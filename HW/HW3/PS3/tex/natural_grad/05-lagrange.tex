\item \subquestionpoints{8} \textbf{Natural Gradient}

Now we move on to calculating the natural gradient. Recall that we want to maximize the log-likelihood by moving only by a fixed $\KL$ distance from the current position. In the previous sub-question we came up with a way to approximate $\KL$ distance with Fisher Information. Now we will set up the constrained optimization problem that will yield the natural gradient update $d$. Let the log-likelihood objective be $\ell(\theta) = \log p(y;\theta)$. Let the $\KL$ distance we want to move by, be some small positive constant $c$. The natural gradient update $d^*$ is
\begin{align*}
d^* &= \arg\max_d \ell(\theta + d) \quad \text{subject to} \quad \KL(p_\theta||p_{\theta + d}) = c & \text{(1)}
\end{align*}

First we note that we can use Taylor approximation on $\ell(\theta + d) \approx \ell(\theta) + d^T\nabla_{\theta'}\ell(\theta')|_{\theta'=\theta}$. Also note that we calculated the Taylor approximation $\KL(p_\theta||p_{\theta+d})$ in the previous subproblem. We shall substitute both these approximations into the above constrained optimization problem.

In order to solve this constrained optimization problem, we employ the \emph{method of Lagrange multipliers}. If you are familiar with Lagrange multipliers, you can proceed directly to solve for $d^*$. If you are not familiar with Lagrange multipliers, here is a simplified introduction. (You may also refer to a slightly more comprehensive introduction in the Convex Optimization section notes, but for the purposes of this problem, the simplified introduction provided here should suffice).

Consider the following constrained optimization problem
$$d^\ast =\arg\max_d f(d) \quad \text{subject to} \quad g(d)=c$$
The function $f$ is the objective function and $g$ is the constraint. We instead optimize the \emph{Lagrangian} $\mathcal{L}(d,\lambda)$, which is defined as

$$\mathcal{L}(d,\lambda) = f(d) - \lambda [ g(d)-c ]$$

with respect to both $d$ and $\lambda$. Here $\lambda \in \R_+$ is called the Lagrange multiplier. In order to optimize the above, we construct the following system of equations:
\begin{align*}
 \nabla_d \mathcal{L}(d,\lambda) &= 0, &\text{(a)} \\
 \nabla_\lambda \mathcal{L}(d,\lambda) &= 0. &\text{(b)}
\end{align*}

So we have two equations (a and b above) with two unknowns ($d$ and $\lambda$), which can be sometimes be solved analytically (in our case, we can).

The following steps guide you through solving the constrained optimization problem:

\begin{itemize}
\item Construct the Lagrangian for the constrained optimization problem (1) with the Taylor approximations substituted in for both the objective and the constraint.

\item Then construct the system of linear equations (like (a) and (b)) from the Lagrangian you obtained.

\item From (a), come up with an expression for $d$ that \emph{involves} $\lambda$.

At this stage we have already found the ``direction'' of the natural gradient $d$, since $\lambda$ is only a positive scaling constant. For most practical purposes, the solution we obtain here is sufficient. This is because we almost always include a learning rate hyperparameter in our optimization algorithms, or perform some kind of a line search for algorithmic stability. This can make the exact calculation of $\lambda$ less critical. Let's call this expression $\tilde{d}$ (involving $\lambda$) as the \emph{unscaled natural gradient}. Clearly state what is $\tilde{d}$ as a function of $\lambda$.

The remaining steps are to figure out the value of the scaling constant $\lambda$ along the direction of $d$, for completeness.

\item Plug in that expression for $d$ into (b). Now we have an equation that has $\lambda$ but not $d$. Come up with an expression for $\lambda$ that does \emph{not include} $d$.

\item Plug that expression for $\lambda$ (without $d$) back into (a). Now we have an equation that has $d$ but not $\lambda$. Come up with an expression for $d$ that does \emph{not include} $\lambda$.
\end{itemize}

The expression for $d$ obtained this way will be the desired natural gradient update $d^*$. Clearly state and highlight your final expression for $d^*$. This expression cannot include $\lambda$.

