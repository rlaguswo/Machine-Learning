\begin{answer}
    Let denote $p(y= 1|x; \phi, \mu_0, \mu_1, \sigma)$ as $p(y=1|x)$
	\begin{equation}
	    p(y= 1|x) = {p(x|y=1)p(y=1)\over p(x|y=1)p(y=1)+p(x|y=0)p(y=0)}
	\end{equation}
	
	\begin{equation}
	    \phi exp(-0.5(x-\mu_1)^{T}\Sigma^{-1}(x-\mu_1))\over \phi exp(-0.5(x-\mu_1)^{T}\Sigma^{-1}(x-mu_1))+ (1-\phi)exp(-0.5(x-\mu_0)^{T}\Sigma^{-1}(x-\mu_o)) 
	\end{equation}
	
	\begin{equation}
	     1\over 1+exp(0.5((x-\mu_1)^{T}\Sigma^{-1}(x-\mu_1) - (x-\mu_0)^{T}\Sigma^{-1}(x-\mu_0)) + log(1-\phi) - log(\phi))
	\end{equation}
	So, we get equation (9) from equation(8).
	
	Since, $(x-\mu_1)^{T}\Sigma^{-1}(x-\mu_1) = x^{T}\Sigma^{-1}x - \mu_1^{T}\Sigma^{-1}x - x^{T}\Sigma^{-1}\mu_1 + \mu_1^{T}\Sigma^{-1}\mu_1$
	
	we get $(x-\mu_1)^{T}\Sigma^{-1}(x-\mu_1) - (x-\mu_0)^{T}\Sigma^{-1}(x-\mu_0) $
	
 $	= (\mu_0-\mu_1)^{T}\Sigma^{-1}x + x^{T}\Sigma^{-1}(\mu_0-\mu_1) +(\mu_1^{T}\Sigma^{-1}\mu_1 - \mu_0^{T}\Sigma^{-1}\mu_0)$
	
	
	Since $(\mu_0-\mu_1)^{T}x$ is only 1x1 matrix
	, and $((\mu_0-\mu_1)^{T}x)^{T} = x^{T}\Sigma^{-1}(\mu_0-\mu_1)$
	
	Thus we got the equation,
	\begin{equation}
	(\mu_0-\mu_1)^{T}\Sigma^{-1}x + x^{T}\Sigma^{-1}(\mu_0-\mu_1 = 2(\mu_0-\mu_1)^{T}\Sigma^{-1}
	\end{equation}
	
    \begin{equation}
	1\over 1+exp(0.5*2(\mu_0-\mu_1)^{T}\Sigma^{-1}x+0.5(\mu_1^{T}\Sigma^{-1}\mu_1 - \mu_0^{T}\Sigma^{-1}\mu_0)+log(1-\phi)-log(\phi))
	\end{equation}
	
	So,
	\begin{equation}
	    -\theta^{T}x = (\mu_0-\mu_1)^{T}\Sigma^{-1}x \Rightarrow \theta = \Sigma^{-1}(\mu_1-\mu_0)
	\end{equation}
	
	\begin{equation}
	    -\theta_0 = log(1-\phi)-log(\phi)+0.5(\mu_1^{T}\Sigma^{-1}\mu_1 - \mu_0^{T}\Sigma^{-1}\mu_0) 
	\end{equation}
	\begin{equation}
	    \Rightarrow \theta_0 = 0.5(\mu_0^{T}\Sigma^{-1}\mu_0 - \mu_1^{T}\Sigma^{-1}\mu_1) + log(\phi) - log(1-\phi)
	\end{equation}
	
	We derived $\theta$ in the form of $\phi$, $\Sigma$, $\mu_0$, and $\mu_1$.
	
	Therefore, 
	\begin{equation}
	p(y = 1\mid x; \phi, \mu_0, \mu_1, \Sigma)
	= \frac{1}{1 + \exp(-(\theta^T x + \theta_0))}
	\end{equation}

	
\end{answer}
