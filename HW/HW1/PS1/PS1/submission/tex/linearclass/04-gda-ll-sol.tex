\begin{answer}
we will find maximum parameters of log likelihood by searching $\nabla l = 0$ for each parameter. 

  \begin{equation*}
      l(\phi,\mu_0, \mu_1, \Sigma) 
      = \sum_{i=1}^\nexp(log(p(x^{i}|y^{i}) + \sum_{i=1}^\nexp(log(p(y^{i})))
  \end{equation*}
  
  \begin{equation*}
      p(y^{i}) = \phi^{y^{i}}(1-\phi)^{y^{i}}
\end{equation*}
\begin{equation*}
      p(x^{i}|y^{i}) = \frac{1}{(2\pi)^{0.5d}|\Sigma|^{0.5}}exp(-0.5(x-\mu_(y^{i}))^{T}\Sigma^{-1}(x-\mu_(y^{i})))
 \end{equation*}
 
 \begin{equation*}
     \nabla_\phi l = 0 = \nabla_\phi \sum_{i=1}^\nexp(log(p(x^{i}|y^{i}) + \nabla_\phi\sum_{i=1}^\nexp(log(p(y^{i}))) = 
\end{equation*}
\begin{equation*}
      \nabla_\phi\sum_{i=1}^\nexp(log(\frac{1}{(2\pi)^{0.5d}|\Sigma|^{0.5}}exp(-0.5(x-\mu_(y^{i}))^{T}\Sigma^{-1}(x-\mu_(y^{i})))) + \nabla_\phi \sum_{i=1}^\nexp log(\phi^{y^{i}}(1-\phi)^{y^{i}}) = 
 \end{equation*} 
 \begin{equation*}
     0 + \sum_{i=1}^\nexp (\frac{y^{i}}{\phi} - \frac{1-y^{i}}{1-\phi}) = 0
 \end{equation*}
 \begin{equation*}
     \sum_{i=1}^\nexp (\frac{y^{i}}{\phi}) = \sum_{i=1}^\nexp( \frac{1-y^{i}}{1-\phi}) \Rightarrow \frac{1}{\phi}\sum_{i=1}^\nexp y^{i} = \frac{1}{1-\phi}(n - \sum_{i=1}^\nexp y^{i}) 
 \end{equation*}
 
 If we denote $\sum_{i=1}^\nexp y^{i}$ as X for the convenience of calculation, we get
 \begin{equation*}
     \frac{x}{\phi} = \frac{n-X}{1-\phi} \Rightarrow X-X\phi = n\phi - X\phi
 \end{equation*}
 Thus
 \begin{equation*}
     \phi = \frac{X}{n} = \frac{1}{n}\sum_{i=1}^\nexp y^{i} =  \frac{1}{n}\sum_{i=1}^\nexp 1\{y^{(i)} = 1\} \\
 \end{equation*}
 
For finding maximum $\mu_y$ for the likelihood function, we need to use $\nabla_\mu_y l = 0$.

First, 
\begin{equation*}
    l = \sum_{i=1}^\nexp(log(p(x^{i}|y^{i})) + \sum_{i=1}^\nexp log(p(y^{i}) = 
\end{equation*}
\begin{equation*}
\sum_{i=1}^\nexp(log(\frac{1}{(2\pi)^{0.5d}|\Sigma|^{0.5}}exp(-0.5(x-\mu_(y^{i}))^{T}\Sigma^{-1}(x-\mu_(y^{i})))) + \sum_{i=1}^\nexp log(\phi^{y^{i}}(1-\phi)^{y^{i}})
\end{equation*}
 \begin{equation*}
     \nabla_\mu_y\sum_{i=1}^\nexp(log(\frac{1}{(2\pi)^{0.5d}|\Sigma|^{0.5}}exp(-0.5(x-\mu_(y^{i}))^{T}\Sigma^{-1}(x-\mu_(y^{i})))) + \nabla_\mu_y \sum_{i=1}^\nexp log(\phi^{y^{i}}(1-\phi)^{y^{i}})
 \end{equation*}
 \begin{equation*}
     0 = \nabla_\mu_y\sum_{i=1}^\nexp (-0.5(x-\mu_y)^{T}\Sigma^{-1}(x-\mu_y)) + 0
 \end{equation*}
\begin{equation*}
    -0.5 *\nabla_\mu_y\sum_{i=1}^\nexp (x^{T}\Sigma^{-1}x-x^{T}\Sigma^{-1}\mu_y - \mu_y^{T}\Sigma^{-1}x + \mu_y^{T}\Sigma^{-1}\mu_y) = 
\end{equation*}
\begin{equation*}
    -0.5*\nabla_\mu_y\sum_{i=1}^\nexp(x^{T}\Sigma^{-1}x- 2\mu_y^{T}\Sigma^{-1}x + \mu_y^{T}\Sigma^{-1}\mu_y) = 
\end{equation*}
 (Since $\mu_y^{T}\Sigma^{-1}x$ is 1x1 matrix, it is equivalent with its transpose $x^{T}\Sigma^{-1}\mu_y$.)
 \begin{equation*}
     \sum_{i=1}^\nexp(\Sigma^{-1}x - \Sigma^{-1}\mu_y) = 0
     \Rightarrow \Sigma \sum_{i=1}^\nexp(\Sigma^{-1}x - \Sigma^{-1}\mu_y) = \sum_{i=1}^\nexp(x - \mu_y) = 0
 \end{equation*}
 Thus,
 \begin{equation*}
     \sum_{i=1}^\nexp x = \sum_{i=1}^\nexp \mu_y
 \end{equation*}
 
 Since $\mu_y$ has different values depending on the y values, we will use indicator function in order to erase the other values which are not relevant to current y value. 
 \begin{equation*}
     \sum_{i=1}^\nexp 1\{y^{(i)} = k\} \\x^{i} = \sum_{i=1}^\nexp 1\{y^{(i)} = k\} \\\mu_y 
 \end{equation*}
 (k = 0 or 1)
 
 
 So, $\mu_y$ is...
 \begin{equation*}
     \frac{\sum_{i=1}^\nexp 1\{y^{(i)} = k\} \\x^{i}}{\sum_{i=1}^\nexp 1\{y^{(i)} = k\} \\} 
 \end{equation*}
 (k = 0 or 1)
 
 From the equation above we can know...
 \begin{equation*}
     \mu_0 = \frac{\sum_{i=1}^\nexp 1\{y^{(i)} = 0\} \\x^{i}}{\sum_{i=1}^\nexp 1\{y^{(i)} = 0\} \\} 
 \end{equation*}
 \begin{equation*}
     \mu_1 = \frac{\sum_{i=1}^\nexp 1\{y^{(i)} = 1\} \\x^{i}}{\sum_{i=1}^\nexp 1\{y^{(i)} = 1\} \\} 
 \end{equation*}
 
 Finally, in order to find the maximum likelihood estimate of $\Sigma$, we need to calculate $\nabla_\Sigma l = 0$
 
 Since $p(y)$ is not consisted of $\Sigma$, we can know that $\nabla_\Sigma\sum_{i=1}^\nexp p(y^{i}) = 0$.
 So, 
 \begin{equation*}
     \nabla_\Sigma l = \nabla_\Sigma\sum_{i=1}^\nexp(-\frac{d*log(2\pi)}{2} -\frac{log(|\Sigma|)}{2} -\frac{(x-\mu_y)^{T}\Sigma^{-1}(x-\mu_y)}{2}) +0 = 0
 \end{equation*}
 \begin{equation*}
     = -\frac{1}{2}\sum_{i=1}^\nexp(0 + (\Sigma^{-1})^{T} - (\Sigma^{-T}(x-\mu_y)(x-\mu_y)^{T}\Sigma^{-T}))
     = -\frac{1}{2}\sum_{i=1}^\nexp(0 + (\Sigma^{-1}) - (\Sigma^{-1}(x-\mu_y)(x-\mu_y)^{T}\Sigma^{-1})) = 0
 \end{equation*}
 
 
 \begin{equation*}
     \Rightarrow -\frac{1}{2}\Sigma(\sum_{i=1}^\nexp(0 + (\Sigma^{-1}) - (\Sigma^{-1}(x-\mu_y)(x-\mu_y)^{T}\Sigma^{-1})))\Sigma = 0
 \end{equation*}
 
 So, we get...
 \begin{equation*}
     \sum_{i=1}^\nexp (\Sigma - (x-\mu_y)(x-\mu_y)^{T}) = 0 \Rightarrow
     n\Sigma -  \sum_{i=1}^\nexp ((x-\mu_y)(x-\mu_y)^{T}) = 0
 \end{equation*}
 Thus, 
 \begin{equation*}
     \Sigma = \frac{1}{n}\sum_{i=1}^\nexp ((x-\mu_y)(x-\mu_y)^{T})
 \end{equation*}
\end{answer}
