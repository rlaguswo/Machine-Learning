\begin{answer}
    Before finding the Hessian of $J(\theta)$, we need to find the gradient of the $J(\theta)$.
    
\begin{equation}
    \nabla J(\theta) = {1\over N}(h_\theta(x^{(i))}-y)x_j
\end{equation}

\begin{equation}
    H = {1\over N}g(\theta^{T}x)(1-g(\theta^{T}x))x_ix_j
      = {1\over N}g(\theta^{T}x)(1-g(\theta^{T}x))xx^{T}
\end{equation}

\begin{equation}
    g(\theta^{T}x) \in R \Rightarrow g(\theta^{T}x)(1-g(\theta^{T}x)) \in R
\end{equation}

\begin{equation}
     g(\theta^{T}x) > 0 \Rightarrow  g(\theta^{T}x)(1-g(\theta^{T}x)) > 0
\end{equation}

\begin{equation}
    {1\over N}g(\theta^{T}x)(1-g(\theta^{T}x)) \Rightarrow C(constant),C \geq 0
\end{equation}

\begin{equation}
    z^{T}Hz = C*z^{T}xx^{T}z = C\sum_i z_ix_i \sum_j z_jx_j
    = C{(\sum_i x_iz_i)}^2 \geq 0
\end{equation}
    Thus, the Hessian of $J(\theta)$ is Positive Semi Definite.
\end{answer}
